\chapter{Design}

\section{Overall System Design}

\subsection{Short description of the main parts of the system}
\begin{itemize}
\item Home
This system will display the buttons required to  access the rest of the program and exit the program
\item Match
This system will display the details of a all the past matches that have been entered into the system. 
\item Player 
This system  will display all the details that have been entered by the user on an individual player.  
\item Team Sheet
This system will display the selected players in the selected formation, it will also have the substitutes listed to one side of the screen. The user will also be able to swap players in and out of the team sheet. Once the one user is happy with the team sheet they will be able to save it to the database for future reference. 
\item Goals
This system will display a list of goals from all past games that have been entered into the system, it will include bas details such as the scorer, the match it was scored in and the quantity.
\item Squad
This system will display a list of all the players in the squad.
\item Add Player
This system will ask for the user to input the basic details of a new player, such as name and position. These details will all be entered into text boxes.
\item Add Goals
This system will ask for the user to select the match and goal scorer from a drop down menu, The user will then have to enter the quantity scored by the selected player in the selected game. 
\item Add Match
This system will ask for the user to input the basic details of a recent match, such as the result and the opposition.

\end{itemize}
\section{User Interface Designs}

\section{Hardware Specification}

\section{Program Structure}

\subsection{Top-down design structure charts}

\subsection{Algorithms in pseudo-code for each data transformation process}

\subsection{Object Diagrams}

\subsection{Class Definitions}

\section{Prototyping}

\section{Definition of Data Requirements}

\subsection{Identification of all data input items}

\subsection{Identification of all data output items}

\subsection{Explanation of how data output items are generated}

\subsection{Data Dictionary}

\subsection{Identification of appropriate storage media}

\section{Database Design}

\subsection{Normalisation}

\subsubsection{ER Diagrams}

\subsubsection{Entity Descriptions}

\subsubsection{1NF to 3NF}

\subsection{SQL Queries}

\section{Security and Integrity of the System and Data}

\subsection{Security and Integrity of Data}

\subsection{System Security}

\section{Validation}

\section{Testing}

